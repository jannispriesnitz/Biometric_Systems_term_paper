\section{Introduction}
Face recognition systems have become one of the most popular biometric authentication methods in the last years. It is based on the fairly unique biometric characteristic of a human face. One of their advantages are the property of a contactless capturing the face images with help of an arbitrary high resolution camera system which highly accepted by the data subjects. In addition to this, the capability of a visual inspection instead of an automatic process is one of the reasons why face recognition is selected as authentication method for biometric passports. 
% due to comparing the biometric data set to the individual 
The basic idea is simply to observe certain properties of the human face, such as the shape of the head or wrinkles and furrows, and place landmarks on characterizing points. 

\todo[inline]{Hier könnte noch kurz aging und posing rein.}

Since 2002 \todo{? cite} face recognition is used as identity confirmation in the electronic 
Machine Readable Travel Document (eMRTD) by the  International Civil Aviation Organisation (ICAO). This means every eMRTD issued by an governmental organisation contains an facial image which has to follow certain properties in order to support the machine based automatic verification. \todo{cite / improove}

In several countries, such as \todo{... Spain?}, it is possible to provide own printed pictures to the issuing organisation. This practise leads to the possibility of processing on the photo and therefore altering the biometric data set stored in the eMRTD. Of course these alterations form a potential attack vector on the Automatic Border Control systems (ABC) \todo{introduce earlier}. A feasible attack would be an alteration in the way that another individual than the one which the passport is issued to is recognized by the ABC or both individuals are recognized as the same person. 

To achieve this the face of the issuing individual and an attacker has to be morphed together. The goal on\todo{whitin?} this process is to provide a morphed photo to the issuing instance which visually nearly identical with the issuer but automatically accepts both, the issuer and the attacker. Having reached this both, the issuer and the attacker are able to show up at the ABC system and both will be accepted. 

\todo[inline]{What are we doing on the topic?}

\subsection*{Outline}
 \todo[inline]{label + ref}
The rest of this paper is organized as follows: In ... we provide some details on the topic of face detection followed by describing the procedure of morphing faces. ... deal with the selected detection algorithm and gives some details on our test setup. Finally in ... the result of test subjects are discussed followed by a conclusion in ... .
 