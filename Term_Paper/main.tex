\documentclass[english]{lni}
% for German use:
%\documentclass{lni}

\IfFileExists{latin1.sty}{\usepackage{latin1}}{\usepackage{isolatin1}}

\usepackage{graphicx}
\usepackage{fancyhdr}
\usepackage{listings} %if lstlistings is used
\usepackage{changepage} %for changing topmargin on first page
\usepackage[figurename=Fig., tablename=Tab., small]{caption}[2008/04/01]
\renewcommand{\lstlistingname}{List.}    % Listingname is now List. 

\usepackage[disable]{todonotes}

% multicollumn
\usepackage{multicol}
% *** SUBFIGURE PACKAGES ***
\usepackage[caption=false, font=normalsize, labelfont=sf, textfont=sf]{subfig}


%Beginning of page count for this paper
\setcounter{page}{1}

%head line settings
\pagestyle{fancy}
\fancyhead{} % clears the settings
\fancyhead[RO]{\small A. Br\"omme, C. Busch, C. Rathgeb and A. Uhl (Eds.): BIOSIG 2016, \linebreak Lecture Notes in Informatics (LNI), Gesellschaft f\"ur Informatik, Bonn 2016 \hspace{5pt} \thepage \hspace{0.05cm}}
\fancyfoot{} % clears footer settings
\renewcommand{\headrulewidth}{0.4pt} %horizontal line below header
\setcounter{footnote}{0}

\author{Jannis Priesnitz\footnote{University of Applied Sciences Darmstadt, Department of Computer Science, Schöfferstraße 3, 64295 Darmstadt \email{jannis.priesnitz@stud.h-da.de}}, Julian Thomae\footnote{University of Applied Sciences Darmstadt, Department of Computer Science, Schöfferstraße 3, 64295 Darmstadt \email{jueliant@gmail.com}}}


\title[Capabilities of face morphing]{Capabilities of automatic and manual face morphing}



\begin{document}
	
	\maketitle
	
	\renewcommand{\refname}{References}
	\setcounter{footnote}{2} %Change to the number of authors for a correct numbering of the foot notes
	
	\pagestyle{fancy}
	\fancyhead{} 
	\fancyhead[RO]{\small $<$firstname lastname [et. al.]$>$(Edt.): $
		<$book title$>$, \linebreak Lecture Notes in Informatics (LNI), 
		Gesellschaft f\"ur Informatik, Bonn $<$Jahr$>$ \hspace{5pt} 
		\thepage \hspace{0.05cm}}
	\fancyfoot{} 
	\renewcommand{\headrulewidth}{0.4pt} 
	\begin{abstract}
		In a common scenario one passport including its biometric features belongs to one person. This and only this person \textit{should be} successfully matched to the biometric picture which is in the persons passport in the situation of an Automatic Border Control (ABC). But what if two or more persons are successfully matched to one passport? With the procedure of morphing faces, it is possible to get a promising acceptance rate for both persons. This work compares an automatic with a manual approach for creating morphed faces. In addition the region, in which an sufficient match rate for both persons is reached, is determined to hide the second face as good as possible from manual inspection. 
		
	\end{abstract}
	\begin{keywords}
		Face morphing \and face detection \and automatic border controls
	\end{keywords}
	
%	\linebreak
	\pagestyle{fancy}
	\fancyhead{} 
	\fancyhead[RO]{\small Capabilities of face morphing \hspace{5pt} 
		\thepage \hspace{0.05cm}}
	\fancyhead[LE]{\hspace{0.05cm}\small \thepage \hspace{5pt} 
		Jannis Priesnitz and Julian Thomae}
	\fancyfoot{} 
	\renewcommand{\headrulewidth}{0.4pt}
	
		\section{Introduction}
Face recognition systems have become one of the most popular biometric authentication methods in the last years. It is based on the fairly unique biometric characteristic of a human face. One of their advantages are the property of a contactless capturing the face images with help of an arbitrary high resolution camera system which highly accepted by the data subjects. In addition to this, the capability of a visual inspection instead of an automatic process is one of the reasons why face recognition is selected as authentication method for biometric passports. 
% due to comparing the biometric data set to the individual 
The basic idea is simply to observe certain properties of the human face, such as the shape of the head or wrinkles and furrows, and place landmarks on characterizing points. 

\todo[inline]{Hier könnte noch kurz aging und posing rein.}

Since 2002 \todo{? cite} face recognition is used as identity confirmation in the electronic 
Machine Readable Travel Document (eMRTD) by the  International Civil Aviation Organisation (ICAO). This means every eMRTD issued by an governmental organisation contains an facial image which has to follow certain properties in order to support the machine based automatic verification. \todo{cite / improove}

In several countries, such as \todo{... Spain?}, it is possible to provide own printed pictures to the issuing organisation. This practise leads to the possibility of processing on the photo and therefore altering the biometric data set stored in the eMRTD. Of course these alterations form a potential attack vector on the Automatic Border Control systems (ABC) \todo{introduce earlier}. A feasible attack would be an alteration in the way that another individual than the one which the passport is issued to is recognized by the ABC or both individuals are recognized as the same person. 

To achieve this the face of the issuing individual and an attacker has to be morphed together. The goal on\todo{whitin?} this process is to provide a morphed photo to the issuing instance which visually nearly identical with the issuer but automatically accepts both, the issuer and the attacker. Having reached this both, the issuer and the attacker are able to show up at the ABC system and both will be accepted. 

\todo[inline]{What are we doing on the topic?}

\subsection*{Outline}
 \todo[inline]{label + ref}
The rest of this paper is organized as follows: In ... we provide some details on the topic of face detection followed by describing the procedure of morphing faces. ... deal with the selected detection algorithm and gives some details on our test setup. Finally in ... the result of test subjects are discussed followed by a conclusion in ... .
 
		%\input{Fundamentals} no longer needed
		\section{Database and selection of test subjects}
As test sample a data set of XXX\todo{how many?} ICAO compliant pictures were given. In order to get promising morph results, a subset of XX \todo{number} pairs of photos were selected for manual morphing \ref{manual_morph} where as the automatic morphing algorithm \ref{automatic_morph} was applied on all data sets. For manual  morphing only pairs with a visually high coincidence are considered because the aceptance rate of the comparison algorithem is expected to be higher. 
In summation XX manual and XXX automatic morphs are issued in this paper. 

\todo[inline]{ICAO conformance beschrieben}
\todo[inline]{FaceDB}

\section{Morphing of Faces}
The main task during the morphing of two pictures is to detect characteristics and place landmarks as an advince for the algorithm. This can be done completely automatic or with support of an user. In this paper both way are discussed.  
\subsection{Basic idea}
\label{percentageMorph}
For every morph there were 15 images created from 0\% of Subject 1 to 100\%, respectively the remaining \% of Person 2. So the are images combined of:
\begin{itemize}
	\item 1. Picture: Person 1 100\% - Person 2 0\%
	\item 2. Picture: Person 1 92,86\% - Person 2 7,14\%
	\item 3. Picture: Person 1 85,71\% - Person 2 14,29\%
	\item 4. Picture: Person 1 78,57\% - Person 2 21,43\%
	\item 5. Picture: Person 1 71,43\% - Person 2 28,57\%
	\item 6. Picture: Person 1 64,29\% - Person 2 35,71\%
	\item 7. Picture: Person 1 57,14\% - Person 2 42,86\%
	\item 8. Picture: Person 1 50,00\% - Person 2 50,00\%
	\item 9. Picture: Person 1 42,86\% - Person 2 57,14\%
	\item 10. Picture: Person 1 35,71\% - Person 2 64,29\%
	\item 11. Picture: Person 1 28,57\% - Person 2 71,43\%
	\item 12. Picture: Person 1 21,43\% - Person 2 78,57\%
	\item 13. Picture: Person 1 14,29\% - Person 2 85,71\%
	\item 14. Picture: Person 1 7,14\% - Person 2 92,86\%
	\item 15. Picture: Person 1 0\% - Person 2 100\%
	\end{itemize}

\todo[inline]{say something on this + make tabular}
\subsection{Automatic morphing}
\label{automatic_morph}


\subsection{Manual morphing}
\label{manual_morph}
In contrast to the automatic face morphing approach, manual morphing is discussed in this section. 

To achieve morphes, the open source software GNU Image Manipulation Software (GIMP) (Version 2.8.16) with the GIMP Animation Package (GAP) (Version 2.6) was selected for this process. Morphing with GAP follows the simple approach of manually placing connected landmarks at characterizing points in both faces. In \ref{fig:manual_morph} two pictures with a setup of landmarks are shown. It can be observed, that the landmarks are placed at characterizing points in both faces, e.g. at the eye browns, lips and nose. The general shape of the face as well as the shape of the head including the hair is also respected. In the example the facial landmarks are close to each other whereas the landmarks describing the shape of the hair are farer apart. 

The selection of characterizing points is based on *erkenntnissen* from earlier works on the topic of automatic face detection, to achieve an optimal morphing result in these sections \todo{regions?} \todo{cite handbook of bio p.60} and \todo{cite these fancy mp4 thingy}. 

\cite{vukadinovic2005fully}
\begin{figure} 
	\centering
	\subfloat[Subject 1]{%
		\includegraphics[width=0.45\linewidth]{Resources/manualmorph01.jpg}}
	\label{1a}\hfill
	\subfloat[Morph no. 5]{%
		\includegraphics[width=0.45\linewidth]{Resources/manualmorph02.jpg}}
	\label{1b}\\
	\caption{Example of two ICAO compliant photos (1a and 1e) and morphs at stage 5 (1b), 15 (1c) and 25 (1d)}
	\label{fig:manual_morph} 
\end{figure}


The algorithm shifts the landmarks from face one to face two. In addition to this the color of the skin is transmitted. 

%\subsubsection*{Morphing setup}
For the test samples 100 - 125 landmarks were placed, depending on the face characteristics. The output contains a sequence of 30 photos which show different stages of the morphing procedure. 
In figure \ref{fig1} a two subjects and three morphing stages (5, 15 and 25) are shown. The visual inspection of \ref{1c} shows biometric features of both subjects whereas \ref{1b} and \ref{1d} has more similarity to the closer subject but also covers features of the other subject.
A manual post production of the morphs is not necessary because potential revealing details, like the interference \todo{?} of the clothes, glasses or hair is not considered as characteri


\subsubsection*{Results}
\todo{compare to automatic results when there.}
\begin{figure} 
	\centering
	\subfloat[Subject 1]{%
		\includegraphics[width=0.45\linewidth]{Resources/p1.jpg}}
	\label{1a}\hfill
	\subfloat[Morph no. 5]{%
		\includegraphics[width=0.45\linewidth]{Resources/m1.jpg}}
	\label{1b}\\
	\subfloat[Morph no. 15]{%
		\includegraphics[width=0.45\linewidth]{Resources/m2.jpg}}
	\label{1c}\hfill
	\subfloat[Morph no. 25]{%
		\includegraphics[width=0.45\linewidth]{Resources/m3.jpg}}
	\label{1d}\hfill
	\subfloat[Subject 2]{%
		\includegraphics[width=0.45\linewidth]{Resources/p2.jpg}}
	\label{1e} 
	\caption{Example of two ICAO compliant photos (1a and 1e) and morphs at stage 5 (1b), 15 (1c) and 25 (1d)}
	\label{fig1} 
\end{figure}

\todo[inline]{Detailed description of the morphs}
		\section{Face detection}
For our experiments a face detection and recognition is needed. All tests were made with the open source software OpenFace, which was initially released in 2015 and up to now appears to be quite active\cite{amos2016openface}.

\subsection*{Detection algorithm} % alternativ Test setup?
As face detection algorithm the open source software OpenFace was selected. OpenFace is based on a neural network with is fully trained and has high confidence rates in the shipped version \cite{baltruvsaitis2016openface}. Because OpenFaces main goal is to detect faces on arbitrary photos, the accuracy level is expected to be higher if it works on ICAO compliant data sets. 

\todo{OBSOLOETE ICAO compliance}

\subsection*{Process of work}
Both of the two source photos and the sequence of 30 morphs is given as input to the OpenFace comparison algorithm. OpenFace computes the match rate of every morph to both of the two photos. 
The expected outcome of comparison algorithm is an almost equal match rate for morph no. 15, where both pictures are represented to $50\%$. The closer the morph gets to one of the original pictures the higher the match rate and the lower to the other picture. 
\todo{mention match rates of sample pictures in fig 1}

\todo[inline]{hier distances abstrakt beschreiben}
\todo[inline]{beschreiben welcher thershold gewählt wurd und warum}
\subsection{Distance and Threshold}
OpenFace determines the similarity level of two subjects by computing the squared Euclidean distance between characterizing points in both faces. The lower the distance is the higher is the similarity of the two subjects. 
For determining if two pictures are from the same subject a threshold is determined with the property that distances lower than the threshold are seen as same subject and distances high as different ones. 
Facing this it is obvious, that subjects with a naturally given high similarity have a lower distance. The original threshold is at 0.999 which reveals different but very similar subject (e.g. twins) as different. Detection capabilities because of morphing are not known up to now.
In \autoref{Threshold} we suggest a threshold for our approaches. 
%The OpenFace results are obtained by computing the squared Euclidean distance on the pairs and labeling pairs under a threshold as being the same person and above the threshold as different people.  The best threshold on the training folds is used as the threshold on the remaining fold.  In nine out of ten experiments, the best threshold is 0.99.  Figure 7 compares OpenFace’s accuracy with other techniques.  Unlike the modern deep neural network-based techniques, the Eigenfaces result uses no outside data.

\todo{NO split?}
		\section{Results}
Resulting from the work are the squared l2 distances from openface. The distance shows the similiarity to the given subjects. A lower distance means the compared two persons are more equal, when the distance is under a given threshold, these two persons are accepted to be the same person and so access is given. The resulting morphed photos were compared to diffrent photos of both subjects, to get a independend distance. For every morph there were 15 images created from 0\% of Subject 1 to 100\%, respectively the remaining \% of Person 2. So the are images combined of:
\begin{itemize}
	\item 1. Picture: Person 1 100\% - Person 2 0\%
	\item 2. Picture: Person 1 92,86\% - Person 2 7,14\%
	\item 3. Picture: Person 1 85,71\% - Person 2 14,29\%
	\item 4. Picture: Person 1 78,57\% - Person 2 21,43\%
	\item 5. Picture: Person 1 71,43\% - Person 2 28,57\%
	\item 6. Picture: Person 1 64,29\% - Person 2 35,71\%
	\item 7. Picture: Person 1 57,14\% - Person 2 42,86\%
	\item 8. Picture: Person 1 50,00\% - Person 2 50,00\%
	\item 9. Picture: Person 1 42,86\% - Person 2 57,14\%
	\item 10. Picture: Person 1 35,71\% - Person 2 64,29\%
	\item 11. Picture: Person 1 28,57\% - Person 2 71,43\%
	\item 12. Picture: Person 1 21,43\% - Person 2 78,57\%
	\item 13. Picture: Person 1 14,29\% - Person 2 85,71\%
	\item 14. Picture: Person 1 7,14\% - Person 2 92,86\%
	\item 15. Picture: Person 1 0\% - Person 2 100\%
\end{itemize}

\subsection{Distances}
All resulting squared l2 distances for the morphed photos of 01-m-002-27 to 01-m-003-24, 01-m-003-24 to 01-m-005-23, 01-m-004-23 to 01-m-005-23, 01-m-010-23 to 01-m-013-23 and 01-m-014-23 to 01-m-016-23 compared to the corresponding compare images are way too many data. So there is as an example the  morphed photo 01-m-002-27 to 01-m-003-24:

%\begin{tabular}{lrrrrrr}
%	Picture & 01-m-002-28.jpg & 01-m-002-29.jpg & 01-m-002-30.jpg & 01-m-003-25.jpg & 01-m-003-26.jpg & 01-m-003-27.jpg \\
%	1 & 0.11916 & 0.07499 & 0.19188 & 1.30874 & 1.16709 & 1.31322\\
%	2 & 0.13701 & 0.06885 & 0.18756 & 1.25716 & 1.10248 & 1.25841\\ 
%	3 & 0.17384 & 0.06523 & 0.19060 & 1.17656 & 1.01354 & 1.17311\\ 
%	4 & 0.22901 & 0.07982 & 0.21253 & 1.08009 & 0.90457 & 1.06856\\ 
%	5 & 0.31766 & 0.12439 & 0.24763 & 0.89989 & 0.70834 & 0.88006\\ 
%	6 & 0.39700 & 0.16766 & 0.30990 & 0.83492 & 0.62518 & 0.81035\\ 
%	7 & 0.50975 & 0.24823 & 0.40167 & 0.72848 & 0.51501 & 0.70676\\ 
%	8 & 0.67400 & 0.39087 & 0.53792 & 0.60985 & 0.39251 & 0.59632\\ 
%	9 & 0.74737 & 0.46525 & 0.58010 & 0.52552 & 0.32146 & 0.50314\\ 
%	10 & 0.94108 & 0.62628 & 0.74969 & 0.40924 & 0.21060 & 0.37129\\ 
%	11 & 1.05918 & 0.76321 & 0.86483 & 0.32472 & 0.13804 & 0.28219\\ 
%	12 & 1.21209 & 0.90143 & 0.99503 & 0.25177 & 0.08833 & 0.20977\\ 
%	13 & 1.25246 & 0.97993 & 1.05583 & 0.19876 & 0.05832 & 0.17236\\ 
%	14 & 1.34758 & 1.07654 & 1.14637 & 0.19252 & 0.04679 & 0.16232\\ 
%	15 & 1.37122 & 1.13813 & 1.18339 & 0.14941 & 0.05522 & 0.13605\\ 
%\end{tabular}

Also shown in 
\begin{figure}[htbp] 
	\centering
	\includegraphics[width=0.4\textwidth]{Resources/result1.jpg}
	\caption{Squared l2 distances (y axis) of morphs from 01-m-002-27 to 01-m-003-24 (with 15 steps ont he x axis) comparing to 01-m-002-28.jpg, 01-m-002-29.jpg, 01-m-002-30.jpg, 01-m-003-25.jpg, 01-m-003-26.jpg and 01-m-003-27.jpg}
	\label{fig}
\end{figure}

The results of the 01-m-002-27 to 01-m-003-24, 01-m-003-24 to 01-m-005-23, 01-m-004-23 to 01-m-005-23, 01-m-010-23 to 01-m-013-23 and 01-m-014-23 to 01-m-016-23 morphs are:
\begin{figure}[htbp] 
	\centering
	\includegraphics[width=0.4\textwidth]{Resources/result1-5.jpg}
	\caption{Squared l2 distances (y axis) of morphs from 01-m-002-27 to 01-m-003-24, 01-m-003-24 to 01-m-005-23, 01-m-004-23 to 01-m-005-23, 01-m-010-23 to 01-m-013-23 and 01-m-014-23 to 01-m-016-23 (with 15 steps on the x axis) comparing to the corresponding compare photos}
	\label{fig:Result1-5}
\end{figure}

For better recognizability the mean value of all the diffrent squared l2 distances is calculated for Person 1 and Person 2. The result is:
\begin{figure}[htbp] 
	\centering
	\includegraphics[width=0.4\textwidth]{Resources/result1-5-mean.jpg}
	\caption{Mean squared l2 distances (y axis) of morphs from 01-m-002-27 to 01-m-003-24, 01-m-003-24 to 01-m-005-23, 01-m-004-23 to 01-m-005-23, 01-m-010-23 to 01-m-013-23 and 01-m-014-23 to 01-m-016-23 (with 15 steps on the x axis) comparing to the corresponding compare photos}
	\label{fig}
\end{figure}
As visible the lowest distance to both persons is at Picture 9 (Person 1 42,86\% - Person 2 57,14\%) wtih a minimal distance of \textbf{0.485}.\\

Openface uses normally a threshold of 0.99, which allows nearly all morphs from Picture 5 to 10 to be successfull acknowledged as shown in \ref{fig:Result1-5}. Only in 3 cases there is the distance way too high to work properly. The compared photos are 01-m-016-24.jpg, 01-m-016-25.jpg and 01-m-016-26.jpg from the same person, so this morph is not working. As a result in 4 out of 5 cases it is possible to morph two subjects to be successfull acknowledged, this makes a sucess chance of 80\%.
		\section{Conclusion}

It takes about 40 minutes to achieve a high quality manual morph with GIMP GAP. 
		\section{Further topics}
\label{FurtherTopics}
\todo[inline]{ungemorpht vorvergleichen}
In this paper we focussed on the capabilities of morphing visually similar subjects together. The selection if "visual similarity" is based on our own personal estimations. An approach to achieve even smaller distances of morphs could be to compare a subject to a plenty of other subjects with OpenFace and then morph the two closest togehter. 
\todo[inline]{Morphen von mehreren bildern}


Another possibility would be to hide two subjects in a morph with the approach of morphing tree photos together. 
\todo[inline]{Visual inspection}


Sooner or later all subjects and passports are visually inspected by a border control officer, so a morph should hold against this verification, too. An interesting question is, until which level humans are able to reveal morphed pictures.
		
	\bibliographystyle{lnig}
	\bibliography{Quellen}
	
\end{document}



