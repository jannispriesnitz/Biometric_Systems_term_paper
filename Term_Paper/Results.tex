\section{Results}
Resulting from the work are the squared l2 distances from openface. The distance shows the similiarity to the given subjects. A lower distance means the compared two persons are more equal, when the distance is under a given threshold, these two persons are accepted to be the same person and so access is given. The resulting morphed photos were compared to diffrent photos of both subjects, to get a independend distance. For every morph there were 15 images created from 0\% of Subject 1 to 100\%, respectively the remaining \% of Person 2. So the are images combined of:
\begin{itemize}
	\item 1. Picture: Person 1 100\% - Person 2 0\%
	\item 2. Picture: Person 1 92,86\% - Person 2 7,14\%
	\item 3. Picture: Person 1 85,71\% - Person 2 14,29\%
	\item 4. Picture: Person 1 78,57\% - Person 2 21,43\%
	\item 5. Picture: Person 1 71,43\% - Person 2 28,57\%
	\item 6. Picture: Person 1 64,29\% - Person 2 35,71\%
	\item 7. Picture: Person 1 57,14\% - Person 2 42,86\%
	\item 8. Picture: Person 1 50,00\% - Person 2 50,00\%
	\item 9. Picture: Person 1 42,86\% - Person 2 57,14\%
	\item 10. Picture: Person 1 35,71\% - Person 2 64,29\%
	\item 11. Picture: Person 1 28,57\% - Person 2 71,43\%
	\item 12. Picture: Person 1 21,43\% - Person 2 78,57\%
	\item 13. Picture: Person 1 14,29\% - Person 2 85,71\%
	\item 14. Picture: Person 1 7,14\% - Person 2 92,86\%
	\item 15. Picture: Person 1 0\% - Person 2 100\%
\end{itemize}

\subsection{Distances}
All resulting squared l2 distances for the morphed photos of 01-m-002-27 to 01-m-003-24, 01-m-003-24 to 01-m-005-23, 01-m-004-23 to 01-m-005-23, 01-m-010-23 to 01-m-013-23 and 01-m-014-23 to 01-m-016-23 compared to the corresponding compare images are way too many data. So there is as an example the  morphed photo 01-m-002-27 to 01-m-003-24:

%\begin{tabular}{lrrrrrr}
%	Picture & 01-m-002-28.jpg & 01-m-002-29.jpg & 01-m-002-30.jpg & 01-m-003-25.jpg & 01-m-003-26.jpg & 01-m-003-27.jpg \\
%	1 & 0.11916 & 0.07499 & 0.19188 & 1.30874 & 1.16709 & 1.31322\\
%	2 & 0.13701 & 0.06885 & 0.18756 & 1.25716 & 1.10248 & 1.25841\\ 
%	3 & 0.17384 & 0.06523 & 0.19060 & 1.17656 & 1.01354 & 1.17311\\ 
%	4 & 0.22901 & 0.07982 & 0.21253 & 1.08009 & 0.90457 & 1.06856\\ 
%	5 & 0.31766 & 0.12439 & 0.24763 & 0.89989 & 0.70834 & 0.88006\\ 
%	6 & 0.39700 & 0.16766 & 0.30990 & 0.83492 & 0.62518 & 0.81035\\ 
%	7 & 0.50975 & 0.24823 & 0.40167 & 0.72848 & 0.51501 & 0.70676\\ 
%	8 & 0.67400 & 0.39087 & 0.53792 & 0.60985 & 0.39251 & 0.59632\\ 
%	9 & 0.74737 & 0.46525 & 0.58010 & 0.52552 & 0.32146 & 0.50314\\ 
%	10 & 0.94108 & 0.62628 & 0.74969 & 0.40924 & 0.21060 & 0.37129\\ 
%	11 & 1.05918 & 0.76321 & 0.86483 & 0.32472 & 0.13804 & 0.28219\\ 
%	12 & 1.21209 & 0.90143 & 0.99503 & 0.25177 & 0.08833 & 0.20977\\ 
%	13 & 1.25246 & 0.97993 & 1.05583 & 0.19876 & 0.05832 & 0.17236\\ 
%	14 & 1.34758 & 1.07654 & 1.14637 & 0.19252 & 0.04679 & 0.16232\\ 
%	15 & 1.37122 & 1.13813 & 1.18339 & 0.14941 & 0.05522 & 0.13605\\ 
%\end{tabular}

Also shown in 
\begin{figure}[htbp] 
	\centering
	\includegraphics[width=0.4\textwidth]{Resources/result1.jpg}
	\caption{Squared l2 distances (y axis) of morphs from 01-m-002-27 to 01-m-003-24 (with 15 steps ont he x axis) comparing to 01-m-002-28.jpg, 01-m-002-29.jpg, 01-m-002-30.jpg, 01-m-003-25.jpg, 01-m-003-26.jpg and 01-m-003-27.jpg}
	\label{fig}
\end{figure}

The results of the 01-m-002-27 to 01-m-003-24, 01-m-003-24 to 01-m-005-23, 01-m-004-23 to 01-m-005-23, 01-m-010-23 to 01-m-013-23 and 01-m-014-23 to 01-m-016-23 morphs are:
\begin{figure}[htbp] 
	\centering
	\includegraphics[width=0.4\textwidth]{Resources/result1-5.jpg}
	\caption{Squared l2 distances (y axis) of morphs from 01-m-002-27 to 01-m-003-24, 01-m-003-24 to 01-m-005-23, 01-m-004-23 to 01-m-005-23, 01-m-010-23 to 01-m-013-23 and 01-m-014-23 to 01-m-016-23 (with 15 steps on the x axis) comparing to the corresponding compare photos}
	\label{fig:Result1-5}
\end{figure}

For better recognizability the mean value of all the diffrent squared l2 distances is calculated for Person 1 and Person 2. The result is:
\begin{figure}[htbp] 
	\centering
	\includegraphics[width=0.4\textwidth]{Resources/result1-5-mean.jpg}
	\caption{Mean squared l2 distances (y axis) of morphs from 01-m-002-27 to 01-m-003-24, 01-m-003-24 to 01-m-005-23, 01-m-004-23 to 01-m-005-23, 01-m-010-23 to 01-m-013-23 and 01-m-014-23 to 01-m-016-23 (with 15 steps on the x axis) comparing to the corresponding compare photos}
	\label{fig}
\end{figure}
As visible the lowest distance to both persons is at Picture 9 (Person 1 42,86\% - Person 2 57,14\%) wtih a minimal distance of \textbf{0.485}.\\

Openface uses normally a threshold of 0.99, which allows nearly all morphs from Picture 5 to 10 to be successfull acknowledged as shown in \ref{fig:Result1-5}. Only in 3 cases there is the distance way too high to work properly. The compared photos are 01-m-016-24.jpg, 01-m-016-25.jpg and 01-m-016-26.jpg from the same person, so this morph is not working. As a result in 4 out of 5 cases it is possible to morph two subjects to be successfull acknowledged, this makes a sucess chance of 80\%.